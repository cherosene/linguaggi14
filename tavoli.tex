%% Progetto di linguaggi di programmazione - Gianluca Grilletti e Giovanni Barbarino
%% Semantica Fully Abstract per PCF


\documentclass{beamer}

\usepackage[utf8]{inputenc}
\usepackage{default}
\usepackage{amssymb}
\usepackage{stmaryrd}
\usepackage{graphicx}
\usepackage{caption}
%\usepackage{subcaption}
\usepackage{tikz}
\usetikzlibrary{arrows,automata}
\newcommand{\eqobs}{\stackrel{\text{obs}}{=}}
\newcommand{\limp}{\mathbin{{-}\mkern-3.5mu{\circ}}}
\newcommand{\tnode}[4]{\node (#1_u) at (#2,#3+0.1) [minimum size=2pt, opacity=0] {#4};
		       \node (#1_d) at (#2,#3-0.1) [minimum size=2pt, opacity=0] {#4};
		       \node (#1) at (#2,#3) [minimum size=2pt] {#4};}


% immagini
\graphicspath{{immagini/}}




\usetheme{Darmstadt}

\title{Un modello fully abstract del PCF}
% \subtitle{Vuoi fare un gioco con me?}
\author{Grilletti Gianluca \and Barbarino Giovanni}
\institute[Unipi]{Università di Pisa}


\begin{document}

\small




%titolo
\begin{frame}
	%\frametitle{Il linguaggio PCF}
	\maketitle
	
\end{frame}

% dag e der
\begin{frame}[t]
\frametitle{strategie $der$ e $\dag$}

Dato un gioco $A$, definiamo la strategia $der$ su $!A\limp A$:
\begin{itemize}
 \item Se $A_i$ è l'$i$-esima componente di $!A$, allora $der_A^i$ si comporta come la strategia copycat su $A_i\limp A \;\cong\; A\limp A$, e non agisce sulle altre componenti 
 \item Dato che le componenti di $!A$ sono equivalenti, avremo che $der_A^i\approx_s der_A^j \: \forall i,j$ , quindi lo indicheremo semplicemente con $der_A$  
\end{itemize}

Data una strategia $\sigma$ su $!A\limp B$, definiamo la strategia $\sigma^{\dag}$ su $!A\limp !B$:


\begin{tikzpicture}
 
 \node (inv) at (0,0) [minimum size=2pt] {};
	 \node (inv2) at (11,-4) [minimum size=2pt] {};
	 
	 \foreach \y in {0,-1.2,-2.4} {
  \foreach \x in {0,2,4} { 
	  \draw [fill=blue, opacity=.5] (\x+1,\y-1.2) rectangle (\x,\y-.2);
	 
	  \node (A\x) at (\x+.5,\y-.7) [minimum size=2pt, scale=1.5] {A};
	  %\node (O) at (\x+.8,\y) [minimum size=2pt, scale=1] {O};
	  }
	  \draw [<->] (A0) -- (A2)  node [above,midway] {P} ;
	  \draw [<->] (A2) -- (A4)  node [above,midway] {P} ;
% 	  \draw [->] (A0) to [loop above] () ;
% 	  \draw [->] (A2) to [loop above] () ;
% 	  \draw [->] (A4) to [loop above] () ;


 \draw [fill=red, opacity=.5] (8,\y-1.2) rectangle (7,\y-.2);
	 \node (B) at (7.5,\y-.7) [minimum size=2pt, scale=1.5] {B};
	 \node (s) at (10,\y-.7) [minimum size=2pt, scale=2] {$\sigma$};
	 \draw [<->] (A4) -- (B)  node [above,midway] {P} ;
	 
	 \ifnum \y<0
	   \draw[<->] (B) -- (7.5,\y+.5)  node [right,midway] {O};
	 \fi
 }
\end{tikzpicture}














\end{frame}


% % prova di implicazione
% \begin{frame}
% \frametitle{Il gioco $A\Rightarrow B$}
% 
% \only<2->{
% 	\[
% 	A \Rightarrow B \quad \equiv \quad !A \limp B
% 	\]
% }
% 
% \onslide<3->{
% \begin{itemize}
%  \item $Gun=A$ con $P_A=\{
%  \only<3-5>{\textcolor{red}{\epsilon}} \only<1-2,6->{\epsilon},
%  \;  \only<6,8>{\textcolor{red}{\underline{pull}}} \only<1-5,7,9->{\underline{pull}}, 
%  \; \only<7>{\textcolor{red}{\underline{pull}\:click}} \only<1-6,8->{\underline{pull}\:click}, 
%  \; \only<9->{\textcolor{red}{\underline{pull}\:bang}} \only<1-8>{\underline{pull}\:bang} 
%  \}$
%  \item $Life=B$ con $P_B=\{
%  \only<3-4>{\textcolor{red}{\epsilon}} \only<1-2,5->{\epsilon},
%  \;  \only<5-9>{\textcolor{red}{*_O}} \only<1-4,10->{*_O}, 
%  \; *_O\checkmark_P,
%  \; *_O1_P,
%  \; \only<10->{\textcolor{red}{*_O2_P}} \only<1-9>{*_O2_P},
%  \; *_O3_P,
%  \; \dots \}$
%  \item<4-> Strategia \textit{Roulette Russa}:
%  
%  $\only<6>{\textcolor{red}{f(*_O)=\underline{pull}_1}} \only<4-5,7->{f(*_O)=\underline{pull}_1}$, 
%  $\quad \only<8>{\textcolor{red}{f(click_n)=\underline{pull}_{n+1}}} \only<4-7,9->{f(click_n)=\underline{pull}_{n+1}} \forall n$,  
%  $\quad \only<10>{\textcolor{red}{f(bang_n)=n_P}} \only<4-9>{f(bang_n)=n_P}$
% \end{itemize}
% }
% 
% 
% 
% \only<2->{ 
% 	\begin{tikzpicture}
% 	 \node (inv) at (0,0) [minimum size=2pt] {};
% 	 \node (inv2) at (11,-4) [minimum size=2pt] {};
% 	 \foreach \x in {0,3,6} { 
% 	  \draw [fill=blue, opacity=.5] (\x+2,-3) rectangle (\x,-1);
% 	 % \node (A) at (\x+1,-2) [minimum size=2pt, scale=2] {A};
% 	  \node (P) at (\x+0.3,-1.3) [minimum size=2pt, scale=1.5] {P};
%  	  \node (O) at (\x+0.3,-2.7) [minimum size=2pt, scale=1.5] {O};
% 	  }
% 	  \node (A) at (7,-2) [minimum size=2pt, scale=2] {A$_1$};
% 	  \node (A) at (4,-2) [minimum size=2pt, scale=2] {A$_2$};
% 	  \node (A) at (1,-2) [minimum size=2pt, scale=2] {A$_3$};
% 	 \draw [fill=red, opacity=.5] (11,-3) rectangle (9,-1);
% 	 \node (B) at (10,-2) [minimum size=2pt, scale=2] {B};
% 	 \node (P) at (9.3,-2.7) [minimum size=2pt, scale=1.5] {P};
%  	 \node (O) at (9.3,-1.3) [minimum size=2pt, scale=1.5] {O};
% 	 
% 	 \only<5->{\node (st1) at (10,-0.5) [minimum size=2pt, scale=1.5] {$*_O$};}
%  	 \only<6->{\node (st2) at (7,-0.5) [minimum size=2pt, scale=1.5] {$\underline{pull}_1$};}
% 	 \only<7->{\node (ri2) at (7,-3.5) [minimum size=2pt, scale=1.5] {$click_1$};}
%  	 \only<8->{\node (st3) at (4,-0.5) [minimum size=2pt, scale=1.5] {$\underline{pull}_2$};}
% 	 \only<9->{\node (ri3) at (4,-3.5) [minimum size=2pt, scale=1.5] {$bang_2$};}
%  	 \only<10->{\node (ri1) at (10,-3.5) [minimum size=2pt, scale=1.5] {$2_P$};}
%  	 
% 	  \only<6->{\draw[->] [out=250,in=280] (st1) to (st2);}
% 	  \only<7->{\draw[->] [out=250,in=30] (st2) to (ri2);}
% 	  \only<8->{\draw[->] [out=80,in=280] (ri2) to (st3);}
% 	  \only<9->{\draw[->] [out=250,in=110] (st3) to (ri3);}
% 	  \only<10->{\draw[->] [out=30,in=150] (ri3) to (ri1);}
% 	  \end{tikzpicture}
% 
% 	}
% 	
% 	
% \end{frame}




\end{document}
