%% Progetto di linguaggi di programmazione - Gianluca Grilletti e Giovanni Barbarino
%% Semantica Fully Abstract per PCF


\documentclass{beamer}

\usepackage[utf8]{inputenc}
\usepackage{default}
\usepackage{amssymb}
\usepackage{stmaryrd}
%\usepackage{graphs}
%\usepackage{tree-dvips}
\usepackage{graphicx}
\usepackage{caption}
\usepackage{subcaption}

\newcommand{\eqobs}{\stackrel{\text{obs}}{=}}
\newcommand{\limp}{\mathbin{{-}\mkern-3.5mu{\circ}}}


% immagini
\graphicspath{{immagini/}}




\usetheme{Darmstadt}

\title{Un modello fully abstract del PCF}
% \subtitle{Vuoi fare un gioco con me?}
\author{Grilletti Gianluca \and Barbarino Giovanni}
\institute[Unipi]{Università di Pisa}


\begin{document}

\small


\section{Il linguaggio PCF}
\subsection{La sintassi}

%titolo
\begin{frame}
	%\frametitle{Il linguaggio PCF}
	\maketitle
	
\end{frame}


% tipi di PCF
\begin{frame}
	\frametitle{I tipi di PCF}
	
	I tipi di PCF sono definiti ricorsivamente a partire dalle seguenti clausole:
	\begin{itemize}
		\item $Nat$ e $Bool$ sono tipi (\emph{i tipi base})
		\item Se $S$ e $T$ sono tipi, $S\times T$ è un tipo
		\item Se $S$ e $T$ sono tipi, $S\rightarrow T$ è un tipo
	\end{itemize}
	
	\begin{example}
		\begin{itemize}
			\item $Nat\times Nat$
			\item $(Nat\times Bool) \rightarrow Bool$
			\item $Nat\rightarrow Nat\rightarrow Nat$ (da intendere $Nat\rightarrow (Nat\rightarrow Nat)$  )
			\item $(Nat \rightarrow Nat) \rightarrow Nat \rightarrow Nat$
		\end{itemize}

	\end{example}

	
\end{frame}


% termini di PCF
\begin{frame}
	
	\begin{block}{Grammatica per generare i termini di PCF}
	
	\begin{align*}
		<nat \_exp> &::= \underline{0} | \underline{1} | \underline{2} | \dots | <nat \_exp> + <nat \_exp> \\
		<bool \_exp> &::= true | false | Eq? <nat \_exp> <nat \_exp> \\
		\quad \\
		<\sigma \rightarrow \tau \_ exp> &::= \lambda (x:\sigma) . <\tau \_exp> \\
		<\sigma \times \tau \_exp> &::= <<\sigma \_exp>, <\tau \_ exp>> \\
		\quad \\
		<\sigma \_exp> &::= <\sigma \_var> |\\
		& if <bool \_exp> then <\sigma \_exp> else <\sigma \_exp> | \\
		<\sigma \_ap&plication> | <\sigma \_ projection> | <\sigma \_ fixed\_point> \\
		<\sigma \_application> &::= <\tau \rightarrow \sigma \_exp><\tau \_exp> \\
		<\sigma \_projection> &::= Proj_1<\sigma \times \tau \_exp> | Proj_2<\tau \times \sigma \_exp> \\
		<\sigma \_fixed \_ point> &::= Y_{\sigma} <\sigma \rightarrow \sigma \_exp>
	\end{align*}
	
	\end{block}
	
	
	
\end{frame}


% operazioni ben tipate (formate)
\begin{frame}
	Con $t:T$ indichiamo che il termine $t$ è di tipo $T$
	\begin{example}
		\begin{itemize}
			\item $(\underline{n} + \underline{m})+ \underline{n} :Nat$
			\item $Eq?(\underline{n})(\underline{m}):Bool$
			\item $<true,\underline{n}>:Bool \times Nat$
			\item $Proj_1 <true,\underline{n}>:Bool$
			\item $\lambda (x:Nat) . x+1 : Nat\rightarrow Nat$ (indichiamolo con $Succ$)
			\item $Succ(\underline{n}):Nat$
			\item \textcolor{red}{$if[Eq?(\underline{n})(\underline{m})]\quad then [\underline{n}]
			\quad else[Succ]$} non è ben formato
			\item $if[Eq?(\underline{n})(\underline{m})]\quad then [\underline{n}]
			\quad else[ Succ(\underline{n}) ] :Nat$
			\item $\lambda(x:Nat).if[Eq?(\underline{0})(x)]\quad then [true]
			\quad else [false]: Nat \rightarrow Bool$ (Indichiamolo con $IsZero$)
			\item $Y [Succ]:Nat$
			\item \textcolor{red}{$Y[IsZero]$} non è ben formato
		\end{itemize}

	\end{example}

	
\end{frame}




% i programmi sono solo di tipo Nat o Bool
\begin{frame}
	
	\frametitle{Programmi}
	
	Un programma di PCF è un termine:
	\begin{itemize}
		\item ben formato
		\item chiuso
		\item di tipo $Nat$ o $Bool$ (tipi \emph{osservabili})
	\end{itemize}

	
	\begin{example}
		\begin{itemize}
			\item $Eq?(\underline{n})(\underline{m}):Bool\quad$ è un programma
			\item $Y [Succ]:Nat\quad$ è un programma
			\item $Succ:Nat\rightarrow Nat\quad$ \emph{non} è un programma (tipo non osservabile)
			\item $x+\underline{n}:Nat\quad$ \emph{non} è un programma (non è chiuso)
		\end{itemize}

	\end{example}
	
\end{frame}

% riduzioni ed astrazioni
\begin{frame}
	
	\frametitle{Semantica operazionale}
	
	Diamo le seguenti regole di riduzione:
	\begin{description}
		\item[add] $\underline{n}+\underline{m} \rightarrow \underline{n+m}$
		\item[Eq?] $Eq?(\underline{n})(\underline{n})\rightarrow true$
		\item $Eq?(\underline{n})(\underline{m})\rightarrow false$ (per $n$ ed $m$ distinti)
		\item[cond] $if[true]\quad then[M]\quad else[N] \rightarrow M$
		\item $if[false]\quad then[M]\quad else[N] \rightarrow N$
		\item[proj] $Proj_1<M,N> \rightarrow M$
		\item $Proj_2<M,N> \rightarrow N$
		\item[$\alpha$] $\lambda (x:\sigma).M \rightarrow \lambda (y:\sigma).[y/x]M$ (con $y$ non libera in $M$)
		\item[$\beta$] $[\lambda(x:\sigma).M](N) \rightarrow [N/x]M$
		\item[$Y$] $Y_{\sigma} \rightarrow \lambda (f:\sigma \rightarrow \sigma).f(Y_{\sigma}f)$
	\end{description}
	
\end{frame}

% Forma Normale
\begin{frame}
	\begin{itemize}
		\item Indichiamo con $\twoheadrightarrow$ la chiusura transitiva della relazione $\rightarrow$
		\item Diciamo che un termine $N$ è in forma normale se non può essere ridotto tramite le regole sopra introdotte
		\item Dato un termine $M$, diciamo che la sua valutazione rispetto alla semantica operazionale è $N$ se
		\begin{itemize}
			\item $N$ è in forma normale
			\item $M \twoheadrightarrow N$
		\end{itemize}
		E lo indichiamo con $Eval(M)=N$
	\end{itemize}
	
	\begin{theorem}[Proprietà di Church-Rosser]
		Se $M \twoheadrightarrow N_1$ e $M \twoheadrightarrow N_2$, allora esiste $P$ tale che $N_1 \twoheadrightarrow P$ e $N_2 \twoheadrightarrow P$
	\end{theorem}
	
	Questo risultato assicura l'unicità della valutazione
	Non sempre però un termine ha una forma normale, in questo caso scriviamo $Eval(M)=undef$
	
\end{frame}

% Contesto e Equivalenza oss
\begin{frame}
	
	\frametitle{Equivalenza osservazionale}
	
	Definiamo un \emph{contesto} come un termine in cui compare un "buco" indicato con $[\;]$
	
	\begin{example}
		$C[\;] \equiv \lambda (x:Nat).x+[\;]$
		
		Porre il termine $\underline{n}$ nel contesto $C[\;]$ significa considerare il termine
		$C[\underline{n}] \equiv \lambda (x:Nat).x+\underline{n}$
	\end{example}
	
	Diciamo che due termini $M$ ed $N$ sono osservazionalmente equivalenti se per ogni contesto $C[\;]$ si ha $Eval(C[M])=Eval(C[N])$
	e lo indichiamo con $M\eqobs N$
	
\end{frame}




\subsection{Proprietà di PCF}
% calcolabilità
\begin{frame}
	
	\frametitle{Espressività di PCF}
	
	Diciamo che una funzione parziale $f:\mathbb{N}\rightarrow \mathbb{N}$ è \emph{calcolabile} se esiste un programma per computer\footnote{Con computer si intende una macchina a registri (URM); idealmente, un computer con infinita memoria} $P$ tale che:
	\begin{itemize}
		\item Se $f(n)=m$, allora il programma $P$ con input $n$ termina con output $m$
		\item Se $f(n)$ non è definita, allora il programma $P$ con input $n$ non termina
	\end{itemize}
	
	
	
	\begin{block}{Teorema della Fermata}
		Non esiste un algoritmo per capire se un generico programma termini o meno
	\end{block}

	
\end{frame}

% indecidibilità della forma normale
\begin{frame}
	
	\begin{block}{Fatto}
		Data una funzione parziale calcolabile $f$, esiste un termine di PCF $t$ tale che
		\begin{itemize}
			\item Se $f(n)=m$, allora $Eval(t(\underline{n}))=\underline{m}$
			\item Se $f(n)$ non è definito, allora $Eval(t(\underline{n}))=undef$
		\end{itemize}

	\end{block}
	
	\begin{block}{Fatto}
		Non esiste un algoritmo per capire se un generico termine di PCF ammetta una forma normale
	\end{block}
	
	\begin{block}{Fatto}
		Non esiste un algoritmo per capire se due termini di PCF siano osservazionalmente equivalenti
	\end{block}

	
\end{frame}



\subsection{Il problema di un modello fully abstract}
% fully abstract
\begin{frame}
	
	\frametitle{Full Abstraction}
	
	\begin{block}{}
	Diciamo che un modello per PCF è Fully Abstract se e solo se per ogni coppia di termini $M$ e $N$:
	\begin{gather*}
		M \eqobs N \Leftrightarrow \llbracket M \rrbracket = \llbracket N \rrbracket
	\end{gather*}
	\end{block}
	
	\begin{block}{}
	Diciamo che un modello per PCF è intensionally fully abstract se:
	\begin{itemize}
		\item È algebrico
		\item Gli elementi compatti sono definibili in PCF
	\end{itemize}
	\end{block}
	
	\begin{block}{Teorema}
		Dato un modello $\mathcal{I}$ intensionally fully abstract, esiste una relazione di equivalenza $\approx$ tale che $\mathcal{E}=\mathcal{I}/ \approx$ sia un modello fully abstract
	\end{block}
	

\end{frame}



% intenzioni di suicidio
\begin{frame}
	
	IL CONTENUTO DI QUESTA SLIDE DIPENDE DA QUANTO VOGLIAMO DIRE ALLA FINE
	
	A questo punto vorremmo un modello per PCF tale che:
	\begin{enumerate}
		\item Sia fully abstract
		\item Il modello sia \emph{definibile} (cioè ogni elemento del modello sia interpretazione di un termine di PCF)
		\item Il modello sia \emph{minimale} (cioè esista una "immersione" in ogni altro modello fully abstract)
	\end{enumerate}
	
	%\begin{block}{Si può chiedere di più? da sistemare}
	%	Si potrebbe richiedere che la denotazione fornisca algoritmi per decidere se due termini siano osservazionalmente equivalenti, almeno per i termini di FinitaryPCF (PCF costruito partendo dal solo tipo base $Bool$); un risultato di Loader ci dice che questo NON è possibile
	%\end{block}
	
	%% Se non sbaglio FinitaryCSS era decidibile; ha senso aggiungere questa cosa perché ne avevamo parlato a lezione

\end{frame}






\section{La categoria dei giochi}


\subsection{giochi e strategie}

% i giochi
\begin{frame}
	\frametitle{I giochi}
	
	Il modello che andremo a considerare si basa sulla \emph{teoria dei giochi}
	
	
	Un gioco è una 4-upla $A=( M_A , \lambda_A , P_A , \approx_A )$ dove:
	\begin{itemize}
	\item $M_A$ è l'insieme delle mosse
	\item $\lambda_A$ è una funzione da $M_A$ all'insieme $\{ O,P\} \times \{Q,A\}$; in particolare:
		\begin{itemize}
		\item $O$ indica il giocatore "opponent" e $P$ il giocatore "player"
		\item $Q$ indica una domanda e $A$ una risposta
		\end{itemize}
	\item Una partita è una stringa di mosse tale che:
		\begin{enumerate}
		\item La prima mossa è di $O$
		\item $P$ e $O$ si alternano
		\item In ogni momento della partita, il numero di risposte deve essere al più uguale al numero di domande (\emph{bracketing condition})
		\end{enumerate}
	\item $P_A$ è un sottoinsieme prefix-closed di partite; chiameremo $P_A$ l'insieme delle partite valide
	\item  $\approx_A$ è una relazione di equivalenza sulle partite valide
	\end{itemize}
	
	
\end{frame}

% strategie
\begin{frame}

	\frametitle{Strategie}
	
	Una strategia $\sigma$ è un insieme di partite di lunghezza pari (l'ultima mossa è di $P$) tali che:
	\begin{itemize}
		\item $\sigma$ è prefix-closed
		\item le strategie sono history free, cioè
		\begin{itemize}
			\item $sab , tac \in \sigma \Rightarrow b=c$
			\item $sab\in \sigma, ta\in P_A \Rightarrow tab\in \sigma$
		\end{itemize}

% 	\item $\sigma$ è prefix-closed
% 	\item $sab , tac \in \sigma \Rightarrow b=c$ (history free)
% 	\item $sab\in \sigma, ta\in P_A \Rightarrow tab\in \sigma$
	\end{itemize}
	
	\begin{exampleblock}{Albero di Gioco}
 
	Prendiamo un gioco, il cui set di mosse $M$ è suddiviso dalla funzione di labelling $\lambda$ in
	  \begin{gather*}
	      M_{QO}=\{a,b,c\}, \quad  M_{AO}=\{g,i\} \\
	      M_{QP}=\{h\},    \quad  M_{AP}=\{d,e,f\}
          \end{gather*}  
	Il set di partite valide $P$, la relazione di equivalenza $\approx$, e le strategie del gioco si possono rappresentare in maniera semplice tramite il \emph{Game Tree}
	\end{exampleblock}
	
	
\end{frame}

% game tree

\begin{frame}

\only<1>{\begin{gather*}
	      M_{QO}=\{a,b,c\}, \quad  M_{AO}=\{g,i\} \\
	      M_{QP}=\{h\},    \quad  M_{AP}=\{d,e,f\}
          \end{gather*}  }

\only<2>{         $s\approx t$ se:
		\begin{itemize}
			\item $s$ e $t$ hanno la stessa etichettatura
			\item per ogni $s'$ e $t'$ sottostringhe iniziali di $s$ e $t$ tali che $|s'|=|t'|$, si ha $s' \approx t'$
			\item se $sa$ è una partita valida, allora esiste $b$ tale che $tb\approx sa$
		\end{itemize}
         } 
\only<3>{

Una strategia $\sigma$ è un insieme di partite di lunghezza pari tali che:
	\begin{itemize}
		\item $\sigma$ è prefix-closed
		\item $sab , tac \in \sigma \Rightarrow b=c$
		\item $sab\in \sigma, ta\in P_A \Rightarrow tab\in \sigma$
	\end{itemize}
	}
	
\only<4>{
	\[
	f_\sigma(x)=
	\begin{cases}
	 d \text{  se } x=a\\
	 / \text{  se } x=b\\
	 h \text{  se } x=c\\
	 d \text{  se } x=g
	\end{cases}
	\qquad
	f_\tau(x)=
	\begin{cases}
	 d \text{  se } x=a\\
	 e \text{  se } x=b\\
	 h \text{  se } x=c\\
	 d \text{  se } x=g
	\end{cases}
	\]
	}
	
\only<5>{	
Estendiamo la relazione $\approx$ alle strategie; poniamo:
	\begin{itemize}
	\item $\underline{ \sigma \preccurlyeq \tau }$ se per ogni $sab \in \sigma$ e $s' \in \tau$, se $sa\approx s'a'$ allora esiste $b'$ tale che $s'a'b' \in \tau$ e $sab\approx s'a'b'$
	\item $\underline{ \sigma \approx \tau \ }; \text{iif} \; \sigma \preccurlyeq \tau \wedge \tau \preccurlyeq \sigma$
	\end{itemize}
	In questo caso, avremo $\sigma \preccurlyeq \tau, \sigma \not\approx \sigma, \tau \approx \tau $  
	}
	
\only<6>{
	\begin{itemize}
		\item $\preccurlyeq$ è un preordine sulle strategie; di conseguenza $\approx$ è una relazione di equivalenza parziale
		\item Nel caso l'equivalenza $\approx$ del gioco sia l'identità, il game tree diventa un albero semplice, e l'ordine tra strategie si può vedere come inclusione di insiemi o tra le funzioni parziali
	\end{itemize}
	}
	
	
	\includegraphics<1>[width=1.5\textwidth, height=0.7\textheight]{immagini/Drawing}
	\includegraphics<2>[width=\textwidth, height=0.7\textheight]{immagini/classi}
	%\includegraphics<3>[width=1.3\textwidth, height=0.6\textheight]{sigma}
	\includegraphics<3-6>[width=\textwidth, height=0.65\textheight]{immagini/sigmatau}

\end{frame}

% tavolo e carte
\begin{frame}
	
	Come rappresentiamo i giochi (il tavolo insomma)
	
	\begin{figure}
		\Huge
		\centering
		\def\svgwidth{0.32\columnwidth}
		\input{immagini/tavolo_a.pdf_tex}
	\end{figure}
	
\end{frame}

% prodotto tensore
\begin{frame}
	
	\frametitle{Il gioco $A \otimes B$}
	
	\begin{itemize}
		\item $M_{A\otimes B}=M_A \coprod M_B$
		\item $\lambda_{A\otimes B}=\lambda_A \coprod \lambda_B$
		\item $P_{A\otimes B}$ sono tutte le partite $s$ tali che:
		\begin{itemize}
			\item $s|_{M_A} \in P_A \wedge s|_{M_B} \in P_B$
			\item Per ogni domanda in $A$, la risposta deve essere in $A$; lo stesso con $B$
		\end{itemize}
		\item $s\approx_{A\otimes B} t \Leftrightarrow s|_A \approx_A t|_A \wedge s|_B \approx_B t|_B \wedge fst(s)=fst(t)$ 
	\end{itemize}
	
	\begin{block}{Proprietà}
		\begin{itemize}
			\item Solamente il giocatore $O$ può cambiare componente di gioco
			\item Il prodotto tensore è associativo
			\item Esiste un elemento neutro $I$
		\end{itemize}
		
	\end{block}
	
	
\end{frame}


\begin{frame}[t]
	
	\begin{tabular}{c c}
		
		%%%%%%% TAVOLO 1
		
		&
		%%%%%%% TAVOLO 2
		
		\\
		
		\begin{minipage}{0.48\textwidth}
			\begin{figure}[t]
				\Large
				\centering
				\def\svgwidth{0.6\textwidth}
				\input{immagini/tavolo_a.pdf_tex}
			\end{figure}
		\end{minipage} &  \begin{minipage}{0.48\textwidth}
			\begin{figure}[t]
				\Large
				\centering
				\def\svgwidth{0.6\textwidth}
				\input{immagini/tavolo_b.pdf_tex}
			\end{figure}
		\end{minipage} \\
		
		%%%%%%% TAVOLO 1
		
		&
		%%%%%%% TAVOLO 2
		
		
	\end{tabular}
	

	
\end{frame}



% gioco implicazione lineare
\begin{frame}
	
	\frametitle{Il gioco $A \limp B$}
	
	\begin{itemize}
		\item $M_{A\limp B}=M_A \coprod M_B$
		\item $\lambda^{QA}_{A\limp B}=\lambda^{QA}_A \coprod \lambda^{QA}_B$
		
		$\lambda^{OP}_{A\limp B}=\overline{\lambda^{OP}_A} \coprod \lambda^{OP}_B$
		\item $P_{A\otimes B}$ sono tutte le partite $s$ tali che:
		\begin{itemize}
			\item $s|_{M_A} \in P_A \wedge s|_{M_B} \in P_B$
			\item Per ogni domanda in $A$, la risposta deve essere in $A$; lo stesso con $B$
		\end{itemize}
		\item $s\approx_{A\otimes B} t \Leftrightarrow s|_A \approx_A t|_A \wedge s|_B \approx_B t|_B \wedge fst(s)=fst(t)$ 
	\end{itemize}
	
	\begin{block}{Proprietà}
		
		\begin{itemize}
			\item Solamente il giocatore $P$ può cambiare componente di gioco
		\end{itemize}
	
	\end{block}
	
\end{frame}



\begin{frame}[t]
	
	\begin{tabular}{c c}
		
		%%%%%%% TAVOLO 1
		
		&
		%%%%%%% TAVOLO 2
		
		\\
		
		\begin{minipage}{0.48\textwidth}
			\begin{figure}[t]
				\Large
				\centering
				\def\svgwidth{0.6\textwidth}
				\input{immagini/tavolo_a_storto.pdf_tex}
			\end{figure}
		\end{minipage} &  \begin{minipage}{0.48\textwidth}
			\begin{figure}[t]
				\Large
				\centering
				\def\svgwidth{0.6\textwidth}
				\input{immagini/tavolo_b.pdf_tex}
			\end{figure}
		\end{minipage} \\
		
		%%%%%%% TAVOLO 1
		
		&
		%%%%%%% TAVOLO 2
		
		
	\end{tabular}
	

	
\end{frame}




% gioco unione
\begin{frame}
	
	\frametitle{Il gioco $A \& B$}
	
	\begin{itemize}
		\item $M_{A\& B}=M_A \coprod M_B$
		\item $\lambda_{A\& B}=\lambda_A \coprod \lambda_B$
		\item $P_{A\& B}=P_A \coprod P_B$
		\item $\approx_{A\& B} \; = \; \approx_A \coprod \approx_B$ 
	\end{itemize}
	
	\begin{block}{Proprietà}
		\begin{itemize}
			\item Una partita di $A\& B$ è giocata su una sola delle due componenti
			\item Ogni strategia di $A\& B$ è unione di una strategia di $A$ e di una strategia di $B$
		\end{itemize}
		
	\end{block}
	
\end{frame}




\begin{frame}[t]
	
	\begin{tabular}{c c}
		
		%%%%%%% TAVOLO 1
		
		&
		%%%%%%% TAVOLO 2
		
		\\
		
		\begin{minipage}{0.48\textwidth}
			\begin{figure}[t]
				\Large
				\centering
				\def\svgwidth{0.6\textwidth}
				\input{immagini/tavolo_a.pdf_tex}
			\end{figure}
		\end{minipage} &  \begin{minipage}{0.48\textwidth}
			\begin{figure}[t]
				\Large
				\centering
				\def\svgwidth{0.6\textwidth}
				\input{immagini/tavolo_b.pdf_tex}
			\end{figure}
		\end{minipage} \\
		
		%%%%%%% TAVOLO 1
		
		&
		%%%%%%% TAVOLO 2
		
		
	\end{tabular}
	

	
\end{frame}






% gioco prodotto tensore infinito
\begin{frame}
	
	\frametitle{Il gioco $!A$}
	
	\begin{itemize}
		\item $M_{!A}=\omega \times M_A$
		\item $\lambda_{!A}(i,a)=\lambda_A(a)$
		\item $s$ è una partita di $P_{!A}$ se e solo se:
		\begin{itemize}
			\item $\forall i\in \omega , s|_i \in P_A$
			\item Se una domanda è nella componente $i$, la sua risposta deve essere nella componente $i$ (\emph{indexed bracketing condition})
		\end{itemize}

		\item $s\approx_{!A} t$ sse esiste $\pi:\omega \rightarrow \omega$ permutazione tale che $s|_i \approx_A t|_{\pi(i)} \wedge (\pi \circ fst)(s)=fst(t)$
	\end{itemize}
	
	\begin{block}{Proprietà}
		\begin{itemize}
			\item Solamente il giocatore $O$ può cambiare componente di gioco
		\end{itemize}
	\end{block}
	
	Nota: concettualmente il gioco $!A$ si comporta come se avessimo infinite copie di $A$ tensorizzate $A\otimes A\otimes A\otimes A\otimes \dots$
	
\end{frame}



\begin{frame}[t]
	
	\begin{tabular}{c c c c}
		
		%%%%%%% TAVOLO 1
		
		&
		%%%%%%% TAVOLO 2
		
		&
		%%%%%%% TAVOLO 3
		
		&
		%%%%%%% TAVOLO 4
		
		\\
		
		\begin{minipage}{0.22\textwidth}
			\begin{figure}[t]
				\large
				\centering
				\def\svgwidth{0.8\textwidth}
				\input{immagini/tavolo_a.pdf_tex}
			\end{figure}
		\end{minipage} &  \begin{minipage}{0.22\textwidth}
			\begin{figure}[t]
				\large
				\centering
				\def\svgwidth{0.8\textwidth}
				\input{immagini/tavolo_a.pdf_tex}
			\end{figure}
		\end{minipage} & \begin{minipage}{0.22\textwidth}
			\begin{figure}[t]
				\large
				\centering
				\def\svgwidth{0.8\textwidth}
				\input{immagini/tavolo_a.pdf_tex}
			\end{figure}
		\end{minipage} & \begin{minipage}{0.22\textwidth}
			\begin{figure}[t]
				\large
				\centering
				\def\svgwidth{0.8\textwidth}
				\input{immagini/tavolo_a.pdf_tex}
			\end{figure}
		\end{minipage} \\
		
		%%%%%%% TAVOLO 1
		
		&
		%%%%%%% TAVOLO 2
		
		&
		%%%%%%% TAVOLO 3
		
		&
		%%%%%%% TAVOLO 4
		
		
	\end{tabular}
	

	
\end{frame}




% esempi di strategie
\begin{frame}

	\frametitle{Alcune strategie}
	
	DOBBIAMO DECIDERE COME SPIEGARE LE STRATEGIE; QUESTE ANDREBBERO DETTE:
	\begin{itemize}
		\item $\sigma ; \tau$ [NO]
		\item $id_{A\limp A}$ (la copy-cat) [SI]
		\item $App:(A \Rightarrow B)\& A \Rightarrow B$ [SI]
	\end{itemize}

\end{frame}



\begin{frame}[t]
	
	\frametitle{La strategia copycat}
	
	\begin{tabular}{c c}
		
		%%%%%%% TAVOLO 1
		
		&
		%%%%%%% TAVOLO 2
		
		\\
		
		\begin{minipage}{0.48\textwidth}
			\begin{figure}[t]
				\Large
				\centering
				\def\svgwidth{0.6\textwidth}
				\input{immagini/tavolo_a_storto.pdf_tex}
			\end{figure}
		\end{minipage} &  \begin{minipage}{0.48\textwidth}
			\begin{figure}[t]
				\Large
				\centering
				\def\svgwidth{0.6\textwidth}
				\input{immagini/tavolo_a.pdf_tex}
			\end{figure}
		\end{minipage} \\
		
		%%%%%%% TAVOLO 1
		
		&
		%%%%%%% TAVOLO 2
		
		
	\end{tabular}
	

	
\end{frame}



% definizione di categoria
% \begin{frame}
% 	
% 	\frametitle{La categoria dei giochi $\mathcal{G}$}
% 	
% 	Una categoria $\mathcal{C}$ è:
% 	\begin{itemize}
% 		\item una collezione di \emph{oggetti} $C_0$
% 		\item dati due oggetti $A,B$, una collezione di \emph{morfismi} (o \emph{frecce}) $A \rightarrow B$
% 		\item dati tre oggetti $A,B,C$, una \emph{legge di composizione $\circ$} tale che:
% 		\begin{itemize}
% 			\item $f: A \rightarrow B \wedge g: B \rightarrow C \Rightarrow g\circ f : A \rightarrow C$
% 			\item $f: A \rightarrow B \wedge g: B \rightarrow C \wedge h: C \rightarrow D \Rightarrow (h\circ g) \circ f = h \circ (g \circ f)$
% 		\end{itemize}
% 		\item dato un oggetto $A$, un morfismo $id_A: A\rightarrow A$ tale che:
% 		\begin{itemize}
% 			\item $f: A \rightarrow B \Rightarrow f \circ id_A = f$
% 			\item $g: B \rightarrow A \Rightarrow id_A \circ g = g$
% 		\end{itemize}
% 
% 	\end{itemize}
% 	
% 	MA QUESTA SLIDE VA MESSA DAVVERO? \\
% 	(Exodd) NO! Non dobbiamo addentrarci nelle categorie, nemmeno di striscio! Citarle neitoremi basta e avanza!
% 	
% \end{frame}

% categoria dei giochi
\begin{frame}
	
	\frametitle{La categoria dei giochi $\mathcal{G}$}
	
	Definiamo $\mathcal{G}$ la categoria tale che:
	\begin{itemize}
		\item $\mathcal{G}_0$ sono i giochi
		\item dati due giochi $A$ e $B$, i morfismi $A\rightarrow B$ sono $\{ \sigma \text{ strategia di } A\limp B | \sigma \approx \sigma \} / \approx$
		\item Date $[\sigma] : A\rightarrow B$ e $[\tau] : B \rightarrow C$, $[\tau] \circ [\sigma] = [\sigma ; \tau]$
	\end{itemize}
	
	In particolare $\mathcal{G}$ è dotata di:
	\begin{itemize}
		\item un oggetto finale ($1$)
		\item è una categoria monoidale (è definito $\otimes$ bifuntore associativo e con elemento neutro)
		\item è una categoria autonoma (per ogni gioco $A$ esiste il suo gioco duale $1 \limp A$)
		\item NON è una categoria cartesiana chiusa (mancano i prodotti)
	\end{itemize}
	
\end{frame}

% categoria di co-Kleisli
\begin{frame}

	Definiamo $K_!(\mathcal{G})$ la categoria di \emph{co-Kleisli} di $\mathcal{G}$ rispetto a $!$; in particolare:
	\begin{itemize}
		\item $K_!(\mathcal{G})_0 = \mathcal{G}_0$
		\item Dati due giochi $A,B$, $Mor_{K_!(\mathcal{G})}(A,B) = Mor_{\mathcal{G}}(!A,B)$
		\item Date due strategie $\sigma$ e $\tau$, $\tau \circ \sigma = \sigma \fatsemi \tau = \sigma ^\dag ; \tau$
		\item Dato un gioco $A$, il morfismo identico è $der_A$
	\end{itemize}

	In particolare $K_!(\mathcal{G})$ è una \emph{categoria cartesiana chiusa}, cioè:
	\begin{itemize}
		\item Dati due oggetti esiste il \emph{prodotto} ($A\& B$)
		\item Esiste un oggetto \emph{finale} ($1$)
		\item Dati due oggetti, esiste l'oggetto esponente ($A \Rightarrow B$ definito come $!A \limp B$; cioé $Mor(A\& B,C) \cong Mor(A,B\Rightarrow C)$)
	\end{itemize}

	SI PUÒ TAGLIARE UN PO' QUESTA? (FORSE!)

\end{frame}

% ppo e ccc razionale
\begin{frame}
	
	\frametitle{order enrichement}
	
	Definiamo un \emph{pointed poset} come un poset con un minimo (generalmente indicato con $\perp$)
	
	Definiamo una categoria cartesiana chiusa $C$ \emph{pointed poset enriched} se:
	\begin{itemize}
		\item Dati due oggetti $A,B$, $(Mor(A,B),\sqsubseteq _{A,B},\perp _{A,B})$ è un pointed poset
		\item Composizione, prodotto e currying sono monotoni
		\item Per ogni $f: A\rightarrow B$, per ogni gioco $C$, $\perp_{B,C} \circ f = \perp _{A,B}$
	\end{itemize}
	
	Definiamo una categoria cartesiana chiusa $C$ \emph{razionale} se:
	\begin{itemize}
		\item è ppo-enriched
		\item per ogni $f: A\times B \rightarrow B$ si ha:
		\begin{itemize}
			\item La catena $(f^{(k)} | k\in \omega)$ definita da $f^{(0)}=\perp _{A,B}$ e $f^{k+1} = f \circ <id_A , f^{(k)}>$ ammette \emph{lub} $f^{\triangledown}$
			\item Dati $g:C\rightarrow A$ e $h:B\rightarrow D$, $g\circ f^\triangledown \circ h = \bigsqcup_{k\in \omega} g \circ f^{(k)} \circ h$
		\end{itemize}

	\end{itemize}
	
\end{frame}




\section{Il modello di Abramsky}

\subsection{Denotazione di PCF}
% I giochi sono un modello di PCF
\begin{frame}
	
	Dato $A$ gioco, date $[\sigma],[\tau]$ classi di strategie di $A$, definiamo
	$[\sigma] \sqsubseteq_A [\tau] \Leftrightarrow \sigma \preccurlyeq_A \tau$ \\
% 	NOTA: con l'equinvalenza banale, non c'è bisogno di parlare di classi
	\begin{block}{Teorema}
		$K_! (\mathcal{G})$ con l'ordine $\sqsubseteq$ è razionale
	\end{block}
	
	\begin{block}{Teorema}
		Sia $C$ una categoria cartesiana chiusa razionale. Si ha che:
		\begin{itemize}
			\item Fissata la denotazione dei tipi base di PCF in $C$ (ogni tipo viene denotato con un oggetto)
			\item Fissata la denotazione delle costanti di PCF in $C$ (ogni termine di tipo $\tau$ viene denotato con un morfismo di $1\rightarrow \llbracket \tau \rrbracket$)
		\end{itemize}
		allora la denotazione può essere estesa a tutti i termini di PCF
		
	\end{block}
	
\end{frame}

% interpretazioni di tipi
\begin{frame}
	
		
	\begin{example}
		\begin{description}
			\item[$Bool$] \begin{itemize}
			              	\item $M_{Bool}=\{*,t,f\}$
			              	\item $\lambda_{Bool}= \{ (*,OQ) ; (t,PA) ; (f,PA) \}$
			              	\item $P_{Bool}= \{ \epsilon , * ,*t, *f \}$
			              	\item $\approx_{Bool} = id_{Bool}$
			              \end{itemize}

			\item[$Nat$] \begin{itemize}
			              	\item $M_{Nat}=\{ * , \underline{0} , \underline{1} , \dots \}$
			              	\item $\lambda_{Nat}= \{ (*,OQ) , (\underline{0},PA) , (\underline{1},PA) , \dots \}$
			              	\item $P_{Nat}= \{ \epsilon , * , *\underline{0} , *\underline{1} , \dots \}$
											\item $\approx_{Nat} = id_{Nat}$
			              \end{itemize}
		\end{description}

	\end{example}	
	DOBBIAMO METTERE UN PAIO DI INTERPRETAZIONI\\
	(Exodd) un paio? Sta tutto qui il difficile!
	
\end{frame}



\subsection{Proprietà del modello intensionale}

% intensional fully abstract
\begin{frame}
	
	\frametitle{Intensional full abstraction}
	
	
	\begin{block}{Teorema}
		
		Per ogni tipo $\tau$ di PCF, posto $T= \llbracket \tau \rrbracket$, si ha che $1 \rightarrow T$ è un dI-domain;
		in particolare $\mathcal{M}(K_! (\mathcal{G}) )$ è un \emph{cpo-based model} algebrico
		
	\end{block}
	
	
	\begin{block}{Teorema}
		
		$\mathcal{M}(K_! (\mathcal{G}) )$ è un modello intensionally fully abstract di PCF
		
	\end{block}
	
	
\end{frame}




\subsection{Il modello estensionale e proprietà}

% Modello Fully Abstract (qui finisce la mia comprensione)
\begin{frame}
	
	\frametitle{Full abstraction}
	
	Definiamo il gioco di \emph{Sierpinsky} $\Sigma$ tale che:
	\begin{itemize}
		\item $M_\Sigma = \{ q,a \}$ dove $\lambda_\Sigma (q)=OQ$ e $\lambda_\Sigma (a)=PA$
		\item $P_\Sigma = \{ \epsilon , q , qa \}$ e $\approx_\Sigma = id_{P_\Sigma}$
	\end{itemize}
	
	Definiamo il preordine $\lesssim_A$ sulle strategie del gioco $A$: $x \lesssim_A y \; \Leftrightarrow \; \forall \alpha \rightarrow \Sigma . x;\alpha \preccurlyeq_\Sigma y;\alpha$
	
	Definiamo $\mathcal{E} = K_!(\mathcal{G}) / \lesssim$, cioè la categoria tale che:
	\begin{itemize}
		\item $\mathcal{E}_0 = K_!(\mathcal{G})_0$
		\item $Mor_{\mathcal{E}}(A,B) = Mor_{K_!(\mathcal{G})}(A,B) / \lesssim_{A\Rightarrow B}$
	\end{itemize}

	
	\begin{block}{Teorema}
		
		$\mathcal{E}$ è un modello fully abstract per PCF
		
	\end{block}
	
	DA SCRIVERE MEGLIO
	
\end{frame}

\subsection{Universalità}

% eh??
\begin{frame}
	
	\frametitle{Universalità}
	
	Definiamo un gioco $A$ \emph{effettivamente dato} se:
	\begin{itemize}
		\item Esiste una mappa $e_A : \omega \rightarrow M_A$ suriettiva; chiamiamo questa funzione codifica
		\item Rispetto alla codifica le seguenti funzioni sono calcolabili:
		\begin{itemize}
			\item $\lambda_A$ (rispetto a qualche codifica di $\{ P,O,Q,A \}$ )
			\item la funzione caratteristica di $P_A$
			\item la funzione caratteristica di $\approx_A$
		\end{itemize}
		
	\end{itemize}
	
	Definiamo una strategia \emph{ricorsiva} se la sua funzione parziale associata è calcolabile
	
	
\end{frame}

% e l'universalità?
\begin{frame}
	
	Definiamo $\mathcal{G}_{rec}$ la categoria dei giochi effettivamente dati con morfismi le strategie ricorsive
	
	\begin{block}{Fatti}
		
		\begin{itemize}
			\item Possono essere definite le categorie $K_!(\mathcal{G}_{rec})$ ed $\mathcal{E}_{rec}$ con ragionamenti analoghi ai precedenti
			\item $\mathcal{E}_{rec}$ è un modello fully abstract per PCF
		\end{itemize}
		
	\end{block}
	
	
	\begin{block}{Universalità}
		
		Ogni strategia di $\mathcal{M}(\mathcal{E}_{rec})$ è definibile in PCF, cioè
		è interpretazione di un termine di PCF
		
	\end{block}
	
	
	(Exodd) Dov'è finita la proprietà di universalità?
	
	
\end{frame}









\end{document}
