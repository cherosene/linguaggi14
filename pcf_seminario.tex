%% Progetto di linguaggi di programmazione - Gianluca Grilletti e Giovanni Barbarino
%% Semantica Fully Abstract per PCF


\documentclass{beamer}

\usepackage[utf8]{inputenc}
\usepackage{default}
\usepackage{amssymb}
\usepackage{stmaryrd}

\newcommand{\eqobs}{\stackrel{\text{obs}}{=}}



\usetheme{Darmstadt}

\begin{document}

\small


\section{Il linguaggio PCF}
\subsection{La sintassi}

\begin{frame}
	\frametitle{Il linguaggio PCF}
	
	PCF (\emph{Programming Computable Functions}) è un linguaggio di programmazione basato sul $\lambda$ calcolo tipato con in aggiunta un operatore $Y$ di punto fisso
	


	
\end{frame}



\begin{frame}
	\frametitle{I tipi di PCF}
	
	I tipi di PCF sono definiti ricorsivamente a partire dalle seguenti clausole:
	\begin{itemize}
		\item $Nat$ e $Bool$ sono tipi (\emph{i tipi base})
		\item Se $S$ e $T$ sono tipi, $S\times T$ è un tipo
		\item Se $S$ e $T$ sono tipi, $S\rightarrow T$ è un tipo
	\end{itemize}
	
	\begin{example}
		\begin{itemize}
			\item $Nat\times Nat$
			\item $(Nat\times Bool) \rightarrow Bool$
			\item $Nat\rightarrow Nat\rightarrow Nat$ (da intendere $Nat\rightarrow (Nat\rightarrow Nat)$  )
			\item $(Nat \rightarrow Nat) \rightarrow Nat \rightarrow Nat$
		\end{itemize}

	\end{example}

	
\end{frame}

% 
% \begin{frame}
% 	\frametitle{I termini di PCF}
% 	
% 	Per definire i termini di PCF dobbiamo dare:
% 	\begin{itemize}
% 		\item I termini dei tipi base (\emph{ground terms})
% 		\item Delle regole per costruire termini più complessi
% 	\end{itemize}
% 	
% 	Indichiamo con $t:T$ che il termine $t$ è di tipo $T$
% 	
% 	\begin{block}{I termini di $Nat$}
% 		\begin{itemize}
% 			\item Per ogni $n\in \mathbb{N}$ esiste una constante $\underline{n}$
% 			\item Dati $M,N:Nat$, allora $N+M:Nat$
% 		\end{itemize}
% 		
% 
% 	\end{block}
% 
% 	
% 	\begin{example}
% 		\begin{itemize}
% 			\item $\underline{n}+\underline{m}:Nat$
% 			\item $(\underline{n}+\underline{m})+\underline{n}:Nat$
% 		\end{itemize}
% 
% 	\end{example}
% 
% 	
% \end{frame}
% 
% 
% \begin{frame}
% 	
% 	\begin{block}{I termini di $Bool$}
% 		\begin{itemize}
% 			\item $true$ e $false$ sono termini di $Bool$
% 			\item Dati $M,N:Nat$, allora $Eq? (M) (N) :Bool$
% 		\end{itemize}
% 		
% 
% 	\end{block}
% 
% 	
% 	\begin{example}
% 		\begin{itemize}
% 			\item $Eq?(\underline{m}) (\underline{n}):Bool$
% 		\end{itemize}
% 
% 	\end{example}
% 	
% \end{frame}
% 


\begin{frame}
	
	\begin{block}{Grammatica per generare i termini di PCF}
	
	\begin{align*}
		<nat \_exp> &::= \underline{0} | \underline{1} | \underline{2} | \dots | <nat \_exp> + <nat \_exp> \\
		<bool \_exp> &::= true | false | Eq? <nat \_exp> <nat \_exp> \\
		\quad \\
		<\sigma \rightarrow \tau \_ exp> &::= \lambda (x:\sigma) . <\tau \_exp> \\
		<\sigma \times \tau \_exp> &::= <<\sigma \_exp>, <\tau \_ exp>> \\
		\quad \\
		<\sigma \_exp> &::= <\sigma \_var> |\\
		& if <bool \_exp> then <\sigma \_exp> else <\sigma \_exp> | \\
		<\sigma \_ap&plication> | <\sigma \_ projection> | <\sigma \_ fixed\_point> \\
		<\sigma \_application> &::= <\tau \rightarrow \sigma \_exp><\tau \_exp> \\
		<\sigma \_projection> &::= Proj_1<\sigma \times \tau \_exp> | Proj_2<\tau \times \sigma \_exp> \\
		<\sigma \_fixed \_ point> &::= Y_{\sigma} <\sigma \rightarrow \sigma \_exp>
	\end{align*}
	
	\end{block}
	
	
	
\end{frame}



\begin{frame}
	
	Con $t:T$ indichiamo che il termine $t$ è di tipo $T$
	
	\begin{example}
		\begin{itemize}
			\item $(\underline{n} + \underline{m})+ \underline{n} :Nat$
			\item $Eq?(\underline{n})(\underline{m}):Bool$
			\item $<true,\underline{n}>:Bool \times Nat$
			\item $Proj_1 <true,\underline{n}>:Bool$
			\item $\lambda (x:Nat) . x+1 : Nat\rightarrow Nat$ (indichiamolo con $Succ$)
			\item $Succ(\underline{n}):Nat$
			\item \textcolor{red}{$if[Eq?(\underline{n})(\underline{m})]\quad then [\underline{n}]
			\quad else[Succ]$} non è ben formato
			\item $if[Eq?(\underline{n})(\underline{m})]\quad then [\underline{n}]
			\quad else[ Succ(\underline{n}) ] :Nat$
			\item $\lambda(x:Nat).if[Eq?(\underline{0})(x)]\quad then [true]
			\quad else [false]: Nat \rightarrow Bool$ (Indichiamolo con $IsZero$)
			\item $Y [Succ]:Nat$
			\item \textcolor{red}{$Y[IsZero]$} non è ben formato
		\end{itemize}

	\end{example}

	
\end{frame}


\begin{frame}

	\frametitle{TODO}
	Regole di riduzione e Semantica operazionale
	\begin{itemize}
		\item cos'è un programma di PCF [Done]
		\item Riduzione non deterministica (e relativa semantica) [Done]
		\item Proprietà di Church-Rosser [Done]
		\item Riduzione left-most [Ma potremmo farne a meno]
	\end{itemize}
	
\end{frame}


\begin{frame}
	
	\frametitle{Programmi}
	
	Un programma di PCF è un termine:
	\begin{itemize}
		\item ben formato
		\item chiuso
		\item di tipo $Nat$ o $Bool$ (tipi \emph{osservabili})
	\end{itemize}

	
	\begin{example}
		\begin{itemize}
			\item $Eq?(\underline{n})(\underline{m}):Bool\quad$ è un programma
			\item $Y [Succ]:Nat\quad$ è un programma
			\item $Succ:Nat\rightarrow Nat\quad$ \emph{non} è un programma (tipo non osservabile)
			\item $x+\underline{n}:Nat\quad$ \emph{non} è un programma (non è chiuso)
		\end{itemize}

	\end{example}
	
\end{frame}


\begin{frame}
	
	\frametitle{Semantica operazionale}
	
	Diamo le seguenti regole di riduzione:
	\begin{description}
		\item[add] $\underline{n}+\underline{m} \rightarrow \underline{n+m}$
		\item[Eq?] $Eq?(\underline{n})(\underline{n})\rightarrow true$
		\item $Eq?(\underline{n})(\underline{m})\rightarrow false$ (per $n$ ed $m$ distinti)
		\item[cond] $if[true]\quad then[M]\quad else[N] \rightarrow M$
		\item $if[false]\quad then[M]\quad else[N] \rightarrow N$
		\item[proj] $Proj_1<M,N> \rightarrow M$
		\item $Proj_2<M,N> \rightarrow N$
		\item[$\alpha$] $\lambda (x:\sigma).M \rightarrow \lambda (y:\sigma).[y/x]M$ (con $y$ non libera in $M$)
		\item[$\beta$] $[\lambda(x:\sigma).M](N) \rightarrow [N/x]M$
		\item[$Y$] $Y_{\sigma} \rightarrow \lambda (f:\sigma \rightarrow \sigma).f(Y_{\sigma}f)$
	\end{description}
	
\end{frame}


\begin{frame}
	\begin{itemize}
		\item Indichiamo con $\twoheadrightarrow$ la chiusura transitiva della relazione $\rightarrow$
		\item Diciamo che un termine $N$ è in forma normale se non può essere ridotto tramite le regole sopra introdotte
		\item Dato un termine $M$, diciamo che la sua valutazione rispetto alla semantica operazionale è $N$ se
		\begin{itemize}
			\item $N$ è in forma normale
			\item $M \twoheadrightarrow N$
		\end{itemize}
		E lo indichiamo con $Eval(M)=N$
	\end{itemize}
	
	\begin{theorem}[Proprietà di Church-Rosser]
		Se $M \twoheadrightarrow N_1$ e $M \twoheadrightarrow N_2$, allora esiste $P$ tale che $N_1 \twoheadrightarrow P$ e $N_2 \twoheadrightarrow P$
	\end{theorem}
	
	Questo risultato assicura l'unicità della valutazione
	Non sempre però un termine ha una forma normale, in questo caso scriviamo $Eval(M)=\perp$
	
\end{frame}


\begin{frame}
	
	\frametitle{Equivalenza osservazionale}
	
	Definiamo un \emph{contesto} come un termine in cui compare un "buco" indicato con $[\;]$
	
	\begin{example}
		$C[\;] \equiv \lambda (x:Nat).x+[\;]$
		
		Porre il termine $\underline{n}$ nel contesto $C[\;]$ significa considerare il termine
		$C[\underline{n}] \equiv \lambda (x:Nat).x+\underline{n}$
	\end{example}
	
	Diciamo che due termini $M$ ed $N$ sono osservazionalmente equivalenti se per ogni contesto $C[\;]$ si ha $Eval(C[M])=Eval(C[N])$
	e lo indichiamo con $M\eqobs N$
	
\end{frame}




\subsection{Proprietà di PCF}

\begin{frame}
	
	\frametitle{Espressività di PCF}
	
	Diciamo che una funzione parziale $f:\mathbb{N}\rightarrow \mathbb{N}$ è \emph{calcolabile} se esiste un programma per computer* $P$ tale che:
	\begin{itemize}
		\item Se $f(n)=m$, allora il programma $P$ con input $n$ termina con output $m$
		\item Se $f(n)$ non è definita, allora il programma $P$ con input $n$ non termina
	\end{itemize}
	
	* Con computer si intende una macchina a registri (URM); idealmente, un computer con infinita memoria
	
	\begin{block}{fatto}
		Non esiste un algoritmo per capire se un generico programma termini o meno
	\end{block}

	
\end{frame}


\begin{frame}
	
	\begin{block}{Fatto}
		Data una funzione parziale calcolabile $f$, esiste un termine di PCF $t$ tale che
		\begin{itemize}
			\item Se $f(n)=m$, allora $Eval(t(\underline{n}))=\underline{m}$
			\item Se $f(n)$ non è definito, allora $Eval(t(\underline{n}))=undef$
		\end{itemize}

	\end{block}
	
	\begin{block}{Fatto}
		Non esiste un algoritmo per capire se, dato un generico termine di PCF questo ammetta una forma normale
	\end{block}
	
	\begin{block}{Fatto}
		Non esiste un algoritmo per capire se due termini di PCF siano osservazionalmente equivalenti
	\end{block}

	
\end{frame}



\subsection{Il problema di un modello fully abstract}

\begin{frame}
	
	\frametitle{Full Abstraction}
	
	\begin{block}{}
	Diciamo che un modello per PCF è Fully Abstract se e solo se per ogni coppia di termini $M$ e $N$:
	\begin{gather*}
		M \eqobs N \Leftrightarrow \llbracket M \rrbracket = \llbracket N \rrbracket
	\end{gather*}
	\end{block}
	
	\begin{block}{}
	Diciamo che un modello per PCF è intensionally fully abstract se:
	\begin{itemize}
		\item È algebrico
		\item Gli elementi compatti sono definibili in PCF
	\end{itemize}
	\end{block}
	
	\begin{block}{Teorema}
		Dato un modello $\mathcal{I}$ intensionally fully abstract, esiste una relazione di equivalenza $\approx$ tale che $\mathcal{E}=\mathcal{I}/ \approx$ sia un modello fully abstract
	\end{block}
	

\end{frame}


\begin{frame}
	
	A questo punto vorremmo un modello per PCF tale che:
	\begin{enumerate}
		\item Sia fully abstract
		\item Il modello sia \emph{definibile} (cioè ogni elemento del modello sia interpretazione di un termine di PCF)
		\item Il modello sia \emph{minimale} (cioè esista una "immersione" in ogni altro modello fully abstract)
	\end{enumerate}
	
	\begin{block}{Si può chiedere di più? da sistemare}
		Si potrebbe richiedere che la denotazione fornisca algoritmi per decidere se due termini siano osservazionalmente equivalenti, almeno per i termini di FinitaryPCF (PCF costruito partendo dal solo tipo base $Bool$); un risultato di Loader ci dice che questo NON è possibile
	\end{block}
	
	%% Se non sbaglio FinitaryCSS era decidibile; ha senso aggiungere questa cosa perché ne avevamo parlato a lezione

\end{frame}





\section{Il modello di Abramsky}

\subsection{La categoria dei giochi}

\begin{frame}
	Piano malefico:
	\begin{itemize}
	\item Definire un gioco e una strategia
	\item Definire i giochi che interessano
		\begin{itemize}
		\item Gioco prodotto tensore
		\item Gioco implicazione lineare
		\item Gioco prodotto
		\item Gioco "of course"
		\end{itemize}
	\item Definire la categoria dei giochi (ergo parlare un po' di TdC)
	\end{itemize}
\end{frame}

\begin{frame}
	\frametitle{I giochi}
	
	Il modello che andremo a considerare si basa sulla \emph{teoria dei giochi}
	
	
	Un gioco è una 4-upla $A=( M_A , \lambda_A , P_A , \approx_A )$ dove:
	\begin{itemize}
	\item $M_A$ è l'insieme delle mosse
	\item $\lambda_A$ è una funzione da $M_A$ all'insieme $\{ O,P\} \times \{Q,A\}$; in particolare:
		\begin{itemize}
		\item $O$ indica il giocatore "opponent" e $P$ il giocatore "player"
		\item $Q$ indica una domanda e $A$ una risposta
		\end{itemize}
	\item Una partita è una stringa di mosse tale che:
		\begin{enumerate}
		\item La prima mossa è di $O$
		\item $P$ e $O$ si alternano
		\item In ogni momento della partita, il numero di risposte deve essere al più uguale al numero di domande (\emph{bracketing condition})
		\end{enumerate}
	\item $P_A$ è un sottoinsieme prefix-closed di partite; chiameremo $P_A$ l'insieme delle partite valide
	\end{itemize}
	
	
\end{frame}

\begin{frame}
	
	\begin{itemize}
	\item $\approx_A$ è una relazione di equivalenza sulle partite valide tale che:
		\begin{itemize}
		\item $s\approx_A t \Rightarrow \lambda^\star _A (s) = \lambda^\star _A (t) $
		\item $s\approx_A t , s' \sqsubseteq s , t' \sqsubseteq t, |s| = |t| \Rightarrow s' \approx_A t'$
		\item $s\approx_A t , sa\in P_A \Rightarrow \exists b . tb \in P_A \wedge sa \approx_A tb$
		\end{itemize}
	DA DIRE IN MANIERA PIÙ UMANA
	\end{itemize}
	
	\begin{example}
		I giochi $Nat$ e $Bool$
	\end{example}	
	
\end{frame}

\begin{frame}

	\frametitle{Strategie}
	
	Una strategia $\sigma$ è un insieme di partite di lunghezza pari (l'ultima mossa è di $P$) tali che:
	\begin{itemize}
	\item $\sigma$ è prefix-closed (da aggiustare)
	\item $sab , tac \in \sigma \Rightarrow b=c$ (history free)
	\item $sab\in \sigma, ta\in P_A \Rightarrow tab\in \sigma$ (history free 2)
	\end{itemize}
	
	Estendiamo la relazione $\approx_A$ alle strategie
	Poniamo:
	\begin{itemize}
	\item $\sigma \preccurlyeq \tau \; \text{iif} \; sab\in \sigma , s' \in \tau , sa \approx_A s'a' \Rightarrow \exists b' . s'a'b' \in \tau \wedge sab \approx_A s'a'b'$
	\item $\sigma \approx \tau \; \text{iif} \; \sigma \preccurlyeq \tau \wedge \tau \preccurlyeq \sigma$
	\end{itemize}

	\begin{block}{Fatto}
		$\preccurlyeq$ è un preordine sulle strategie; di conseguenza $\approx_A$ è una relazione di equivalenza parziale
	\end{block}
	
\end{frame}

\begin{frame}
	
	\begin{example}
		le strategie di $Nat$ e $Bool$
	\end{example}
	
\end{frame}

\begin{frame}
	
	Come rappresentiamo i giochi (il tavolo insomma)
	
\end{frame}

\begin{frame}
	
	\frametitle{Il gioco $A \otimes B$}
	
\end{frame}








\subsection{Il modello e l'order-enrichement}

\begin{frame}
	Piano malefico:
	\begin{itemize}
	\item	Mettere un po' di basi di TdC
	\item Sparare i CANNONI di TdC
	\item Definire $K_! (\mathcal{G})$
	\item Definire l'order enrichement su $K_! (\mathcal{G})$ e sparare un po' di proprietà (modello razionale)
	\item Dire perché è un modello (ergo interpretare i termini)
	\item Costruire una macchina che computi i costrutti sopra, li analizzi, poi li scarti e vada a sfracellarsi contro un muro
	\end{itemize}
\end{frame}


\subsection{Proprietà del modello intensionale}
\subsection{Il modello estensionale e proprietà}
\subsection{Universalità}




\end{document}
